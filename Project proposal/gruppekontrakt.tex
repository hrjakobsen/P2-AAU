\documentclass[a4paper,12pt,hidelinks]{article}
\usepackage[pdftex]{graphicx} 
\usepackage{type1cm}
\usepackage{a4}
\usepackage{lastpage}
\usepackage{graphicx,hyperref,amsmath,bm,url}
\usepackage[numbers]{natbib}
\usepackage{microtype,todonotes}
\usepackage{a4}
\usepackage[compact,small]{titlesec}
\usepackage[utf8]{inputenc}
\usepackage{placeins}
\usepackage{type1cm}
\usepackage{a4}
\usepackage{lastpage}
\usepackage{multirow}
\usepackage{lscape}
\usepackage{listings} 
\usepackage{subcaption}
\usepackage{pdfpages}
\usepackage[toc,page]{appendix}
\usepackage[ddmmyyyy]{datetime}
\clubpenalty = 10000
\widowpenalty = 10000
\usepackage[T1]{fontenc}
\graphicspath{ {../Billeder/} }
\usepackage{tikz}
\usetikzlibrary{calc}
\usetikzlibrary{shapes}
\usepackage[labelfont=bf]{caption}
\renewcommand{\figurename}{\textbf{Figur}}
\renewcommand{\contentsname}{Indholdfortegnelse}
\renewcommand\appendixtocname{Appendikser}
\renewcommand\appendixpagename{Appendikser}
\usepackage[nottoc,notlof,notlot]{tocbibind} 
\renewcommand\bibname{Referencer}
\renewcommand{\tablename}{Tabel}
\renewcommand{\lstlistingname}{Kodestykke}
\newcommand{\heading}[1]{\paragraph*{#1}\mbox{}}


\makeatletter
\newdimen\@myBoxHeight%
\newdimen\@myBoxDepth%
\newdimen\@myBoxWidth%
\newdimen\@myBoxSize%
\newcommand{\SquareBox}[2][]{%
    \settoheight{\@myBoxHeight}{#2}% Record height of box
    \settodepth{\@myBoxDepth}{#2}% Record depth of box
    \settowidth{\@myBoxWidth}{#2}% Record width of box
    \pgfmathsetlength{\@myBoxSize}{max(\@myBoxWidth,(\@myBoxHeight+\@myBoxDepth))}%
    \tikz \node [shape=rectangle, shape aspect=1,draw=red,inner sep=2\pgflinewidth, minimum size=\@myBoxSize,#1] {#2};%
}%
\makeatother
\newcommand*{\captionsource}[2]{%
  \caption[{#1}]{%
    #1%
    \\\hspace{\linewidth}%
    \textbf{Kilde:} #2%
  }%
}

\newcommand{\boxedtext}[1]{\vspace{1cm}
\centerline{ \fbox{\begin{minipage}{.8\textwidth}
``#1''
\end{minipage}}}
\vspace{1cm}}

\lstdefinestyle{customc}{
  belowcaptionskip=1\baselineskip,
  breaklines=true,
  frame=l,
  xleftmargin=5mm,
  numbers=left,                 
  captionpos=b,
  language=C,
  showstringspaces=false,
  basicstyle=\footnotesize\ttfamily,
  keywordstyle=\bfseries\color{green!40!black},
  commentstyle=\itshape\color{purple!40!black},
  identifierstyle=\color{blue},
  stringstyle=\color{orange},
}


\lstset{escapechar={\%,\#},
       style=customc,
       postbreak=\raisebox{0ex}[0ex][0ex]{\ensuremath{\color{red}\hookrightarrow\space}}}
\lstset{literate=
  {á}{{\'a}}1 {é}{{\'e}}1 {í}{{\'i}}1 {ó}{{\'o}}1 {ú}{{\'u}}1
  {Á}{{\'A}}1 {É}{{\'E}}1 {Í}{{\'I}}1 {Ó}{{\'O}}1 {Ú}{{\'U}}1
  {à}{{\`a}}1 {è}{{\`e}}1 {ì}{{\`i}}1 {ò}{{\`o}}1 {ù}{{\`u}}1
  {À}{{\`A}}1 {È}{{\'E}}1 {Ì}{{\`I}}1 {Ò}{{\`O}}1 {Ù}{{\`U}}1
  {ä}{{\"a}}1 {ë}{{\"e}}1 {ï}{{\"i}}1 {ö}{{\"o}}1 {ü}{{\"u}}1
  {Ä}{{\"A}}1 {Ë}{{\"E}}1 {Ï}{{\"I}}1 {Ö}{{\"O}}1 {Ü}{{\"U}}1
  {â}{{\^a}}1 {ê}{{\^e}}1 {î}{{\^i}}1 {ô}{{\^o}}1 {û}{{\^u}}1
  {Â}{{\^A}}1 {Ê}{{\^E}}1 {Î}{{\^I}}1 {Ô}{{\^O}}1 {Û}{{\^U}}1
  {œ}{{\oe}}1 {Œ}{{\OE}}1 {æ}{{\ae}}1 {Æ}{{\AE}}1 {ß}{{\ss}}1
  {ű}{{\H{u}}}1 {Ű}{{\H{U}}}1 {ő}{{\H{o}}}1 {Ő}{{\H{O}}}1
  {ç}{{\c c}}1 {Ç}{{\c C}}1 {ø}{{\o}}1 {å}{{\r a}}1 {Å}{{\r A}}1
  {€}{{\EUR}}1 {£}{{\pounds}}1
}
\usepackage[ddmmyyyy]{datetime}

\newcommand{\group}{DAT2-A423}

\setcounter{secnumdepth}{4}

\begin{document}
	\title{Project Proposal}
	\author{\group}
	\maketitle	
	\section*{The Problem in Context}
Digital steganography is the art of concealing messages within forms of non-concealed data. This does not necessarily have to be text, it could also be other media, such as images, sound, or videos.

There are several problems related to steganography in the modern world of information technology. No matter if you are an activist, a politician, or a terrorist, there is often a need to send messages in a way that makes them difficult for others to read. Typically this involves some form of encryption of varying strength, but any form of encrypted text or other data can arouse suspicion. What if it was possible to send this data in a way where interceptors would not even know that a secret message was conveyed? 

Steganography can be implemented by anyone, both people with good intentions and those with bad. This raises the ethical question, of whether such a system for concealing information should be developed.

	\section*{Relation to Computer Science}
This project, which relates to steganography and its potential use on social media, requires working with groups from other studies. The fact that a software solution needs to be developed in parallel with other teams, means that the use of object-oriented programming becomes essential. This is primarily due to the vastness of the project, which requires separate modules to communicate. Encapsulation of data becomes important, when other teams are dependent on the input and output of the developed modules.
	
Saving data is an integral part of digital steganography. Knowing how each file format stores information is essential in manipulating the data stream when concealing a message, in either form of text or other media. 
It is also fundamental to the project to understand how social networking sites compress media uploaded by users.

An understanding of algorithmics will allow proper implementation of relevant algorithms, though will also make it possible to design new algorithms to conceal data. Discrete mathematics provides the means of calculating the complexity of the aforementioned algorithms through big-O notation.

	\section*{Initiating Problem}
What obstacles arise when attempting to send hidden data across social media and which algorithms can be used to circumvent these?

	\section*{Fulfilment of the Curriculum}
Being able to create a model that can be used in relation to the development of a programme is an essential part of the P2 project. 

By developing a well-structured C\# programme of substantial size, we will be able to understand and use the syntax of an object oriented programming language, per the curriculum.

To find out if the programme fulfils the requirements of the problem we will perform a number of systematic tests.

Additionally, we are going to be able to describe, reflect upon, and analyse the experiences we have attained through our problem orientated learning. This is strengthened by working with students across studies.

 

\end{document} 