\section{Problemformulering}
\begin{frame}{Problemformulering}{}
	\begin{itemize}
		\item How do we modify the graph-theoretic approach for steganography described in section 2.9 to work with JPEG images where the data is hidden in the DCT coefficients without significant visual changes, and how do we implement this in an object-oriented programming language?
	\end{itemize}
\end{frame}
\section{Encoder/Decoder}
\begin{frame}{Encoder/Decoder}{}
	\begin{itemize}
		\item Graph-Theoretic-Approach
		\item Afkodning af Huffmanelementer
		\item Afkodning af den hemmelige besked
	\end{itemize}
		\resizebox{\textwidth}{!}{%
			\begin{tabular}{|c|c|c|} \hline
				\label{tab:huffmanCodes}
				Values & Category & Bit representation \\ \hline
				$0$ & $0$ & - \\
				$-1,1$ & $1$ & $0,1$ \\
				$-3, -2, 2, 3$ & $2$ & $00, 01, 10, 11$ \\
				$-7, -6, -5, -4, 4, 5, 6, 7$ & $3$ & $000, 001, 010, 011, 100, 101, 110, 111$ \\
				$\vdots$ & $\vdots$ & $\vdots$ \\
				$-32767, -32766 \ldots, 32766, 32767$ & $16$ & $0000000000000000 \ldots 1111111111111111$ \\ \hline
			\end{tabular}
		}
\end{frame}

%%%%%%

\section{Udvikling af Encode/Decode}
\begin{frame}{Udvikling af Encode/Decode}{}
	\begin{enumerate}
		\item Teamwork
		\item Fejlfinding
		\item Tidsudnytning
	\end{enumerate}
\end{frame}

%%%%%%

