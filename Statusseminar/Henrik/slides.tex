%Klynge, Confines, Problemformulering, Fremtidig (Phase two!), Afslutning

\begin{frame}{Cluster}{}
	\begin{itemize}
		\item Working other studies
        \begin{itemize}
            \item Mathemathics
            \item Internet Technology and Computer Systems (ITC)
        \end{itemize}
		\item Varying objectives
        \begin{itemize}
            \item \textb{Mathemathics} - Graph Theory 
            \item \textb{ITC} - Two-way steganographic system
        \end{itemize}
		\item
	\end{itemize}
\end{frame}

%Arbejde med 2 grupper fra 2 studier

%   Matematik 

%   Internet Teknologi og Computer Systemer (ITC)

%Forskellige studieordninger -> forskellige mål

%   Mat -> Graf teori

%   ITC -> Udvikle et system der kan kommunikere over netværk og tillader steganografi

\begin{frame}{Confines of the problem}{}
	\begin{itemize}
		\item Least Significant Bit (LSB)
        \begin{itemize}
            \item Steganalysis
            \item JPEG
        \end{itemize}
		\item Graph Theoretical approach
        \begin{itemize}
            \item Curriculum
            \item Cluster group
        \end{itemize}
		\item
	\end{itemize}
\end{frame}

%Det mest oplagte er LSB

%For let at opdage
	
%	Virker ikke godt med JPEG
		
%		JPEG er mest udbredte

%Valgt at arbejde med den graf teoretiske metode beskrevet af Hetzl og Mützel
%Gode muligheder for at opfylde målene i studieordningen:

%	Brug af objekt orienteret sprog til effektivt at modelere og manipulere en graf 

%	Tillader arbejde på tværs af studierne


\begin{frame}{Problem statement}{}
How do we modify the graph-theoretic approach
for steganography described in section 2.13 to work
with JPEG images where the data is hidden in the
DCT coefficients without significant visual changes,
and how do implement this in an object-oriented
language. Furthermore, how do we represent and
work with the mathematical concept of a graph in
an object-oriented programming language.
\end{frame}

%Find et “navn” til metoden 

%Fjern den sidste del (Furthermore….)

\begin{frame}{Future plans}{}
	\begin{itemize}
		\item Cluster group
        \begin{itemize}
            \item \textb{Mathemathics} - Algorithm 
            \item \textb{ITC} - Application Programming Interface
        \end{itemize}
		\item Varying objectives
        \begin{itemize}
            \item Graph Theory 
            \item Two-way steganographic system
        \end{itemize}
		\item
	\end{itemize}
\end{frame}

%Fremtidigt samarbejde med klynge gruppen:

%	Vi bruger en grådig algoritme først 
%	Håber så at matematikerne finder en bedre løsning

%	Vi udvikler vores program med en veldokumenteret
%	API, så de har mulighed for at bygge videre på programmet
