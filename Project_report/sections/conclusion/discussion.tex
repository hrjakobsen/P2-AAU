\section{Discussion}
We started out by using the relatively simple LSB-method for hiding data in an image.
Specifically we saved an image inside of another PNG-image.
While the differences in the image with and without the message-image were hardly noticeable to the human eye, we did find that comparing colour histograms of the two images revealed the fact that changes had been made.
In an effort to diminish this problem, the graph-theoretical approach we implemented attempted to move pixels around in the image, rather than change their values.
This meant that colour histograms of the image before and after the message had been encoded should look very similar, almost identical.
Any changes would come from the SMALL?!?! (25-40%????) percentage of the quantized DCT values used in the encoding that we did not find any interchangeable values for.
When this happened we were forced to change the individual values, which led to a change in the colour composition of the image.
Using this method we expected the change to be small enough that it would not immediately draw attention to the image, if it was being subjected to a colour histogram.

An entirely different way of looking for changes in an image is by looking at the Euclidean distance of them.
This can be done by calculating the distance between the colour values of each pixel in two equally sized images.
The sum of all these distances can be used as measure of the change from one image to the other.
We used this same method for determining what changes were imposed on images of different size and format, when uploaded to various social media and image-sharing websites.

Comparing these results directly with the ones obtained from the LSB-method would be folly.
While the LSB-method would certainly incur a smaller difference in the Euclidean distance it would not neccesarily make it any better.
First of all using this method required that the user was in possesion of both the original and the stego image.
This would indeed be possible if the parties sharing secret messages in the images, where using images they found on the internet and tampered with them.
They could completely foil this risk of being compromised then, by simply using images that they took themselves.
The other reason for the differences being larger is due to how the JPEG-file format is encoded.
Using the LSB of every pixel there is only the potential to change the value of each pixel by one.
Using the graph-theoretical approach with an M value of four, means that each individual value can be changed by four. 
This change is applied after the quantization step to avoid losing the data again.
This in turn means that the small change of up to four is multiplied by the quantization table, when the image is shown.
Bla bla something something green is less than the other