\subsection{Conclusion}
Digital steganography has proved to be a substantial task and working with this topic has given some hurdles. Through the ages people have sent concealed messages to each other, and now with the help of digital steganography it has become a lot easier to get confidential messages to their goal without interception.

In our problem analysis, we came to the conclusion that it was not as important to try and get messages across social media, but instead more important to get messages across in the most discrete and undetectable way possible. We discovered that by using a LSB method, this wasn't possible, so therefore we had to try something more advanced, which we found in graph theory.
 
In relation to our initial thoughts on hiding data in a cover message, we cannot encode as much information as we initially thought possible. It is also slower than the simpler LSB method, which we had used when introducing ourselves to the topic. However, a positive with the graph theoretic approach is that our level of encoding was not as visible in the colour histograms, as it was when concealing a message using the LSB method.

Using an object-oriented programming language, made it easier and less time consuming for us to create a user interface and help benefit another group (ITC) by means of using an interface which they could implement in their own programme. Working with ITC, made us more aware that our programme could be used on other platforms, and this would have been made a great deal more difficult without the use of an object-oriented language.

As described in The Confines of the Problem section, the grounds for working with the method implemented, was to help improve our understanding and help both groups with creating better projects. This did not happen quite as planned, due to insufficient communication. Thankfully this was not catastrophic as there had also been made an agreement that the groups should be able to work together individually and this was how the course of the project went.

We have created a programme, that can encode messages discretely, and this is something that we desired when we chose this method. It is not as fast as we had hoped, nor is there an option for encoding a large amount of data, but it is, however, undetectable to the human eye and our tests with colour histograms.

This means, that it is possible for us to share images with concealed messages without raising suspicion, which is what we started out with.