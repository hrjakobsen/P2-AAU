\subsection{Conclusion}
Digital steganography has proved to be a substantial task and working with this topic has given some hurdles.
 
In relation to our initial thoughts on hiding data in a cover message, we cannot encode as much information as we initially thought possible. It is also slower than the simpler LSB method. However, a positive with the graph theoretic approach is that our level of encoding was not as visible in the colour histograms, as it was when concealing a message using the LSB method.

Using an object-oriented programming language, made it possible for us to create a user interface and help benefit another group (ITC). Working with ITC, made us more aware that our programme could be used on other platforms, and this would have been a great deal more difficult by not writing the programme in an object-oriented language. 

As described in The Confines of the Problem section, the grounds for working with the method implemented, was to help improve our understanding and help both groups with creating better projects. This did not happen quite as planned, due to insufficient communication. Thankfully this was not catastrophic as there had also been made an agreement that the groups should be able to work together individually and this was how the course of the project went.