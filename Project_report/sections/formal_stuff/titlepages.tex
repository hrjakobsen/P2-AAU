\pdfbookmark[0]{English title page}{label:titlepage_en}
\aautitlepage{%
  \englishprojectinfo{
    Steganography and its use in things %title
  }{%
    Programming and Problem Solving %theme
  }{%
    Spring Semester 2016 %project period
  }{%
    DAT2-A423 % project group
  }{%
    %list of group members
    Henrik Herbst Sørensen \\
    Jacob Askløf Svenningsen\\
    Jakob Meldgaard Kjær\\
    Leo Johannesen Mohr\\
    Mathias Steen Jakobsen\\ 
    Søren Madsen\\
    Theresa Krogh-Walker
  }{%
    %list of supervisors
    Kurt Nørmark
  }{%
    2 % number of printed copies
  }{%
    \today % date of completion
  }%
}{%department and address
  \textbf{Department of Computer Science}\\
  Aalborg University\\
  \href{http://www.cs.aau.dk/}{http://www.cs.aau.dk/}
}{% the abstract
  This report examines the use of steganography for hiding information in JPEG images. 

  It starts out with a contextual analysis of the topic. Who uses steganography? How? Why? Afterwards we do a technical analysis on the most commonly used image format, JPEG, and discuss different methods of embedding information into those. 

  The second part of the report examines how graph theory can be used for embedding data into the images, based on an article from Hetzl and Mutzel. Finally we describe our own implementation of a JPEG steganographic encoder.
}

\cleardoublepage