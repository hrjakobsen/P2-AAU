\pdfbookmark[0]{English title page}{label:titlepage_en}
\aautitlepage{%
  \englishprojectinfo{
    Hiding in Plain Sight - Using a Graph-theoretical Approach to Steganography for hiding data in JPEG images %title
  }{%
    Programming and Problem Solving %theme
  }{%
    Spring Semester 2016 %project period
  }{%
    DAT2-A423 % project group
  }{%
    %list of group members
    Henrik Herbst Sørensen \\
    Jacob Askløf Svenningsen\\
    Jakob Meldgaard Kjær\\
    Leo Johannesen Mohr\\
    Mathias Steen Jakobsen\\ 
    Søren Madsen\\
    Theresa Krogh-Walker
  }{%
    %list of supervisors
    Kurt Nørmark
  }{%
    \today % date of completion
  }%
}{%department and address
  \textbf{Department of Computer Science}\\
  Aalborg University\\
  \href{http://www.cs.aau.dk/}{http://www.cs.aau.dk/}
}{% the abstract
  This report contains an analysis of the problem with Steganography and how it can be detected through Steganalysis. 
  It'll also give some insight in something more contextual, such as if it can be used in Social Media by rebels who lives under a tyrannical regime and whom wishes to communicate with one another, without being detected and risk being executed.
   During the report, some steganographic experiments has been conducted, and the image format known as JPEG, which is by far, the most used format when it comes to sharing images on the internet, has been thoroughly analysed, and the programme which has been developed during the writing of this report, has been designed to work on that very format. 
   The programme is able to embed a message secretly within the image to make it as difficult to detect by the human eye as possible, using Graph Theory from the branch of Mathematics known as Discrete Mathematics. 
}

\cleardoublepage