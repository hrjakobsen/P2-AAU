\pdfbookmark[0]{English title page}{label:titlepage_en}
\aautitlepage{%
  \englishprojectinfo{
    Hiding in Plain Sight - Using Graph Theory to Hide Data in JPEG Images %title
  }{%
    Programming and Problem Solving %theme
  }{%
    Spring Semester 2016 %project period
  }{%
    DAT2-A423 % project group
  }{%
    %list of group members
    Henrik Herbst Sørensen \\
    Jacob Askløf Svenningsen\\
    Jakob Meldgaard Kjær\\
    Leo Johannesen Mohr\\
    Mathias Steen Jakobsen\\ 
    Søren Madsen\\
    Theresa Krogh-Walker
  }{%
    %list of supervisors
    Kurt Nørmark
  }{%
    \today % date of completion
  }%
}{%department and address
  \textbf{Department of Computer Science}\\
  Aalborg University\\
  \href{http://www.cs.aau.dk/}{http://www.cs.aau.dk/}
}{% the abstract
  This report contains an analysis of the problems with steganography and how hidden messages can be detected through steganalysis.

  It also gives insights into the uses of steganography, such as who benefits from it and why there is a need for concealed messages and thereby put it into context.
  
  The report contains some steganographic experiments that have been conducted and also consists of an analysis of the image format known as JPEG. These investigations led to the creation of the final programme, which is developed to work on this very format during the writing of this report. 
  
  The programme is able to embed a message secretly within an image to make it as difficult as possible to detect by the human eye, using the branch of Mathematics known as graph theory.
}
\cleardoublepage