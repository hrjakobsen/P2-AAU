\section{State of the Art}
\label{sec:SOTA}
Different forms of steganography have been invented, either because there was an actual need to send a confidential message, or because it has been seen as a challenge by someone who has an interest in the field.
In this section some of the tools already available for performing digital steganography will be mentioned and examined in the hopes of gaining a greater understanding of them.
This could potentially give an idea of what these tools could be missing or otherwise do wrong in a real-life context. 

\begin{description}
	\item[EZStego] is a piece of Java-based software relying on the principle of manipulating the least significant bits of pixels in an image file (here 		GIF and PICT) for storing another image inside it \citep{E-Council|Press2015}.
	As a result, the colour palette is rearranged, but this is done with the changes being barely, or not at all, visible to the human eye.
	
%	\item MP3stego takes steganography to audio files and makes it possible to hide information in MP3 files. 
%	During the compression process, the data is first compressed, encrypted and then hidden in the MP3 bit stream. 
%	The concealment is done in the Layer III encoding process in the ''inner\_loop'' where the input data is first quantized and the quantizer step size 	%	is then increased until the data can be coded with the available number of bits. 
%	Another loop checks that the distortions caused by quantization do not exceed a defined threshold. 
%	A variable contains the number of bits used for scale-factors and Huffman code data in the MP3 bit stream. 
%	The bits are encoded as its parity by changing the end loop condition of the inner loop. 
%	Only randomly chosen values of the variable are modified\citep{MP3Stego}. 

	\item[JSteg-Jpeg] has its focus on JPEG steganography, as the name suggests.
	The programme reads an image file and a secret message, which are then combined into a JPEG image. 
	It utilises the fact that JPEG compression is split into two stages, the lossy stage using DCT and quantization to compress the image data, and the			lossless stage, using Huffmann encoding to further compress the data.
	The secret message needs to be hidden in the image data between the two stages, since it would otherwise be lost in the lossy stage.
	The rounding process in the quantized DCT coefficients are also modulated. 
	JSteg-Jpeg allows for adjustment of the quality factor of the JPEG compression as this together with the size of the embedded message determines the 		degradation of the quality of the final image \citep{ImageProcessingFrankY}.
	
	\item[OpenPuff] is a freeware steganography and watermarking tool by Cosimo Oliboni and is a more modern approach to steganography, first released in 		2004, last updated July 2012. 
	It lets the user hide data by splitting it into several carrier files forming a so-called carrier chain. 
	The carrier files supported include image files like JPG and PNG, audio files like MP3 and WAV, and video files like 3GP and MP4.
	The programme implements up to five layers of security with the first four layers being encryption, scrambling, whitening (mixing scrambled data 			with noise) and encoding the data. 
	Finally the data is injected in the carrier files \citep{Oliboni2012}.
	The principle behind deniable encryption, is that the existence of an encrypted file or message cannot be proven \citep{Schneier2008}.
	This same principle is extended to steganography with OpenPuff, since the existence of a plaintext message cannot be proven.
	As so, OpenPuff not only hides data, but secures it with 256 bit cryptography.
\end{description}

This is just a small selection of the tools available. 
Most other programmes utilise simple methods similar to some of the examples and choose to deal with only simple file formats like BMP or GIF. 
This tells us that dealing with formats that implement compression is a more complex task that requires research.
It has also become clear that very few solutions take into account, the difficulties of sharing steganographic material online.
This makes compression of steganography very interesting to work with. 
It should be mentioned that several programmes do not share the specifics about the techniques used.