% -*- root: ../../DAT2-A423_Project_Report.tex -*-
\section{State of the Art}
\label{sec:SOTA}
Different forms of steganography have been invented, either because there was an actual need to send a confidential message, or because it has been seen as a challenge by someone who has an interest in the field.
In this section some of the tools already available for performing digital steganography will be mentioned and examined in the hopes of gaining a greater understanding of them.
This could potentially give an idea of what these tools could be missing or otherwise do wrong in a real-life context.

\begin{description}
	\item[EzStego] is a Java-based steganography tool for embedding information in GIF-files.
	GIF images are limited to a palette of 256 colours per pixel, and EzStego takes advantage of this to embed information into the pixels.
	A GIF palette is usually sorted by luminance, but this is not ideal, since two colours with similar luminance could be very different.
	EzStego sorts a GIF palette in a way such that two adjacent colours are very similar, which means that if the colours are switched, it is unlikely that a human would notice.
	The embedding of data in EzStego works by switching colours if the bit to be embedded is '1', and leaving the colours if it is '0'.
	This will create an image that looks a lot like the original image, and it is done without changing the file size.
	It is, however, possible to extract the hidden data if the created stego-image is compared pixel-by-pixel to the original image \citep{Westfeld2000}.

	\item[JSteg-Jpeg] has its focus on JPEG steganography, as the name suggests.
	The programme reads an image file and a secret message, which are then combined into a JPEG image.
	It utilises the fact that JPEG compression is split into two stages, the lossy stage using DCT and quantization to compress the image data, and the			lossless stage, using Huffmann encoding to further compress the data.
	The secret message needs to be hidden in the image data between the two stages, since it would otherwise be lost in the lossy stage.
	The rounding process in the quantized DCT coefficients are also modulated.
	JSteg-Jpeg allows for adjustment of the quality factor of the JPEG compression as this together with the size of the embedded message determines the 		degradation of the quality of the final image \citep{ImageProcessingFrankY}.
	
	\item[OpenPuff] is a freeware steganography and watermarking tool by Cosimo Oliboni and is a more modern approach to steganography, first released in 		2004, last updated July 2012.
	It lets the user hide data by splitting it into several carrier files forming a so-called carrier chain.
	The carrier files supported include image files like JPEG and PNG, audio files like MP3 and WAV, and video files like 3GP and MP4.
	The programme implements up to five layers of security with the first four layers being encryption, scrambling, whitening (mixing scrambled data with noise) and encoding the data.
	Finally the data is injected in the carrier files \citep{Oliboni2012}.
	The principle behind deniable encryption, is that the existence of an encrypted file or message cannot be proven \citep{Schneier2008}.
	This same principle is extended to steganography with OpenPuff, since the existence of a plaintext message cannot be proven.
	As so, OpenPuff not only hides data, but secures it with 256 bit cryptography.
\end{description}

\noindent This is just a small selection of the tools available.
Most other programmes utilise simple methods similar to some of the examples and choose to deal with only simple file formats like BMP or GIF.
It should be mentioned that several programmes do not share the specifics about the techniques used.
