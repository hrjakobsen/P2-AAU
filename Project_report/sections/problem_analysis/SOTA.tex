\section{State of the Art}
The art of steganography is one that has been performed and challenged by many throughout time. Many ways of steganography has been invented, either because there was an actual need to send a confidential message, or because it has been seen as a fun and interesting challenge by someone with an interest and often a certain skill set. In this section some of the tools already available for performing digital steganography will be put to light and examined in the hope of gaining greater knowledge of already existing methods and to hopefully get an idea of what these tools could be missing or doing wrong in a real-life context. 

\begin{itemize}
	\item EZStego is a piece of Java-based software relying on the principle of manipulating the least significant bits of pixels in an image file (here GIF and PICT) for storing another image inside it. \citep{E-Council|Press2015} As a result the colour palette is rearranged, but this is done with the changes barely or not at all being visible to the human eye. 
	
	\item MP3stego takes steganography to audio files and makes it possible to hide information in MP3 files. During the compression process the data is first compressed, encrypted and then hidden in the MP3 bit stream. The hiding itself is done in the Layer III encoding process in the ''inner\_loop'' where the input data is first quantized and the quantiser step size then is increased until the data can be coded with the available number of bits. Another loop checks that the distortions caused by quantization do not exceed a defined threshold. A variable contains the number of bits used for scalefactors and Huffman code data in the MP3 bit stream. The bits are encoded as its parity by changing the end loop condition of the inner loop. Only randomly chosen values of the variable are modified. \citep{MP3Stego}
	
	\item JSteg-Jpeg, as the name suggests, has its focus on JPEG steganography. The programme reads multiple image formats and takes an image file and a secret message that it combines and saves as a JPEG image. It utilizes that the format is split into two stages, the lossy stage using DCT and a quantization step to compress the image data, and the lossless stage using Huffmann coding to further compress. This splitting allow for hiding of the secret message into the image data between the two stages. Also the rounding process in the quantized DCT coefficients are modulated. JSteg-Jpeg allows for adjustment of the quality factor of the JPEG compression as this together with the size of the embedded message determines the degradation of the quality of the final image. \citep{ImageProcessingFrankY}
\end{itemize}

This is just a very small selection of the tools available. Most other programmes utilize similarly simple methods and choose to deal with only simple file formats like BMP or GIF. This is telling us that dealing with formats that implement compression is indeed an excessive task that requires its research. 
It should be mentioned that several programmes do not describe the specific techniques used.