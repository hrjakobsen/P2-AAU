% -*- root: ../../DAT2-A423_Project_Report.tex -*-
\section{The Confines of the Problem}
As described in section SOCIAL MEDIA CRAP social media use compression as a means of saving server space, due to the vast amounts of images being shared every day. 
Even though we will not focus on social media we are still going to use the knowledge acquired in section IMAGE COMPRESSION CRAP about the compression of images. 
This is mainly due to the fact that JPEG is the most commonly used file format, and therefore garners less attention than other file formats.
It also allows the ITC-group to save space in their system, if the files sent back and forth are compressed.

As described in section MAGIC SECTION we have decided to work with the graph-theoretical approach described in section \ref{sec:graphtheory}.


\vspace{5mm}
\begin{centering}
	\begin{tcolorbox}[center title, title=Problem Statement, width=.8\textwidth]
How do we modify the graph-theoretic approach for steganography described in section \ref{sec:graphtheory} to work with JPEG images where the data is hidden in the DCT coefficients without significant visual changes, and how do implement this in an object-oriented language?
	\end{tcolorbox}
\end{centering}

This problem statement allows us to fulfil the curriculum requirements. By answering the question
How can we represent and work with the mathematical concept of a graph in an object-oriented programming language?


Throughout the entire project we will also be using object-oriented programming to understand algorithms by continuously developing smaller prototypes and experimental programmes.
Some of these smaller programmes will be integrated in the final programme.
