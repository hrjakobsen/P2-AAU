% -*- root: ../../DAT2-A423_Project_Report.tex -*-
\section{The Confines of the Problem}
We have decided to work with the graph-theoretical approach. 
We have chosen this based on the ground of it being a topic that covers many of our requirements of this semester. 
Furthermore this also allows us to work closely with the cluster-groups, where the mathematicians are working closely with this approach. 
This will let us implement their work, into our programme and thereby improving the final result.

\vspace{5mm}
\begin{centering}
	\begin{tcolorbox}[center title, title=Problem Statement, width=.7\textwidth]
		How do we implement the graph-theoretical algorithm in an object-oriented programming language, and how do we modify it to work with JPEG images?
	\end{tcolorbox}
\end{centering}