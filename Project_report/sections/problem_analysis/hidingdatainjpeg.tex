% -*- root: ../../DAT2-A423_Project_Report.tex -*-

\section{Hiding Data in JPEG images}

\begin{itemize}
	\item Simple Methods
	\begin{itemize}
		\item \textbf{Comments}
		It is possible to store plain text in a JPEG-file in a comment marker (indicated by the hex-value FFFE).
		Comments are usually used to store copyright information, but it can contain any information in hex bytes.
		Comments are ignored by decoders, but when the image is decoded, it is very easy to see which information succeeds the comment marker.

		\item \textbf{After EOI}
		The EOI marker (hex-value FFD9) marks the End of Image, and any data following that is ignored by the decoder.
		It is therefore possible to store information after the EOI marker in much the same way as the comment marker.

		\item \textbf{Progressive bits}
		The SOS marker (Start of Scan, hex-value FFDA) precedes 6 bytes of information, where only three of those are needed in sequential (and baseline) JPEG-compression.
		This means that it is possible to store a very small amount of data when encoding baseline JPEG.
		It may also be harder for a human to notice any hidden information in a SOS-section of a JPEG-image than in a comment or after an EOI-marker.

		\item \textbf{APP$_n$ markers}
		Application markers (APP$_0-15$, hex-values FFE0-FFEF) are used for application-specific information.
		Apart from APP$_0$, which is used by the JFIF-format, any application can create one or more APP-markers to store information.
		These markers can be used to store secret information, though, as with comments, it is very easy for a human to see.
	\end{itemize}
	\item More Challenging Methods
	\begin{itemize}
		\item \textbf{LSB in Quantization tables}

		It is possible to hide information within the least significant bit of a quantization table without making anyone bat an eye.
		The difference in altering each value by one, is such a negligible change that nobody will notice.
		However if this is done, we'll only get up to 128 bits, which we can change, which is a mere 16 bytes, or 16 letters, but still enough to list something like a certain time and place.
		As an example: ``Basisbaren 12am'', is 15 letters.
		\item \textbf{Template from Huffman Tables}

		The Huffman tables are very important to produce a compressed jpeg file, and each jpeg has them.
		There exist ones which generally works well, but to get the optimal solution, it's necessary to analyse the image you want encoded, and then create Huffman tables based on that analysis.
		It's thus also possible to create Huffman tables where some of the values represents a letter, and then provide the people you wish to see the encoded message with a template for the Huffman tables you're using, telling them in which order to read the pixels for the hidden image to show itself.
		\item \textbf{LSB on JPEG thumbnail}

		The Jpeg standard also defines a thumbnail, which can also act as a cover image for a hidden image, or text, by doing something similar to what we did previously at \ref{sec:lsb-implementation}.
		Doing this will give us many more bytes to work on, compared to the two previously mentioned suggestions.
		Even for a 64x64 thumbnail, we'll be able to encode 1536 letters if only the least significant bit is changed, significantly more data than hiding the message within the quantization tables.
	\end{itemize}
	\item Complicated Methods
	\begin{itemize}
		\item \textbf{A Graph-Theoretic Approach to Steganography}
		
		Stefan Hetzl and Petra Mutzel defines a method where graph theory is used for embedding data into bitmap images\citep{hetzl_2005}. Used on a bitmap image, the algorithm finds which pairs of pixels needs to be interchanged, to be able to store data hidden in the image. This process is in many ways superior to methods such as LSB, because no pixels are actually changed while using this approach, pixels are merely interchanged. This makes it much more difficult to use statistical analysis on the image, to figure out if an image contains embedded data. With JPEG images however, pixels are not set separately, and two pixels cannot be interchanged very easily. Instead this method can be used on the quantization tables instead, and embed the data into those tables.

		\item \textbf{A Reversible Data Hiding Scheme for JPEG Images}
		
		Qiming Li et al. proposes a method for embedding data into the DCT coefficients\citep{Li2010}, a method much more targeted JPEG images than the before mentioned. The process is split into three parts. The selecting algorithm which selects a subset of the AC-DCT coefficients, for storing information. The embedding process which embeds the actual data into the image, and lastly the decoder which reverses the process, and retracts the hidden data from the image. Because of the way JPEG encoding works, as described in section \ref{sec:jpegStudy}, small changes in the DCT coefficients or the quantization, will not lead to visible distortion of the JPEG image.

	\end{itemize}
\end{itemize}
