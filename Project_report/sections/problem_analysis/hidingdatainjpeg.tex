% -*- root: ../../DAT2-A423_Project_Report.tex -*-

\section{Hiding Data in JPEG images}

Based on the information on JPEG in section \ref{sec:jpegStudy} and the programmes discussed in section \ref{sec:SOTA} it has been possible to explore various options in regards to concealing data in JPEG images.
The following are simpler methods of this, where the data is simply ignored by the decoder, and therefore the image appears unchanged.

\begin{description}
	\item[Comments] ---
	It is possible to store plain text in a JPEG-file in a comment marker (bytes 0xFFFE).
	Comments are usually used to store copyright information, but any information can be placed in the segment.
	Comments are ignored by decoders, but if the file is inspected, the marker can be clearly seen.

	\item[After EOI] ---
	The End of Image (EOI) marker (bytes 0xFFD9), and any data following that is ignored by the decoder.
	It is therefore possible to store information after the EOI marker, in much the same way as the comment marker.

	\item[Progressive bits]	---
	The Start of Scan (SOS) marker (bytes 0xFFDA) precedes 6 bytes of information, where only three of those are needed in sequential (and baseline) JPEG-compression.
	This means that it is possible to store a very small amount of data when encoding baseline JPEG.
	It may also be harder for a human to notice any hidden information in the SOS-section of a JPEG-image than in a comment or after an EOI-marker.

	\item[APP$_n$ markers] ---
	Application markers (APP$_0-$APP$_{15}$, bytes 0xFFE0-0xFFEF) are used for application-specific information.
	Apart from APP$_0$, which is used by the JFIF-format, any application can create one or more APP-markers to store information.
	These markers can be used to store secret information, though, as with comments, it is very easy for a human to see.
\end{description}

\noindent The next methods that will be described are based on our experiments with LSB in section \ref{sec:lsb-implementation} and from the study of JPEG in section \ref{sec:jpegStudy}.

\begin{description}
	\item[LSB in Quantization tables] ---
	It is possible to hide information within the least significant bit of a quantization table without arousing suspicion.
	However if this is done, we will only get up to 128 bits that we can change, which is 16 bytes, or 16 characters, but still enough to list 			something like a certain time and place.
	As an example: ``Basisbaren 12am'\textbackslash0' '', is 16 characters.
	
	\item[Template from Huffman Tables] ---
	The Huffman tables are very important in producing a compressed JPEG file.
	Some exist that generally work well, but to get the optimal solution, it is necessary to analyse the image you want encoded, and then create Huffman 		tables based on that analysis.
	Thus, it is also possible to create Huffman tables where some of the values represent a character, and then provide the people you wish to see the 			encoded	message with a template for the Huffman tables you are using, telling them in which order to read the pixels for the hidden image to show 			itself.
	
	\item[LSB on JPEG thumbnail] ---
	The JPEG standard also defines a thumbnail, which can also act as a cover image for a hidden image, or text, by doing something similar to what we previously did using the LSB method in section \ref{sec:lsb-implementation}.
	Doing this will give us many more bytes to work on, compared to the two previously mentioned suggestions.
	Even for a 64x64 thumbnail, we will be able to encode 1536 characters if only the least significant bit is changed. 1536 characters is significantly more than the 16, which can be hidden within the quantization tables.
\end{description}

\noindent Lastly we have two more complicated methods based on two separate articles about steganography in image files.  

\begin{description}
	\item[A Reversible Data Hiding Scheme for JPEG Images] ---		
	Qiming Li et al. proposes a method for embedding data into the DCT coefficients \citep{Li2010}.
	The process is split into three parts. The selecting algorithm which selects a subset of the AC-DCT coefficients, for storing information. 
	The embedding process which embeds the actual data into the image, and lastly the decoder which reverses the process, and retracts the hidden data from 	the image. 
	Because of the way JPEG encoding works, as described in section \ref{sec:jpegStudy}, small changes in the DCT coefficients or the quantization, will 		not lead to visible distortion of the JPEG image.
	
	\item[A Graph-Theoretic Approach to Steganography] ---
	Stefan Hetzl and Petra Mutzel define a method where graph theory is used for data into a list of values \citep{hetzl_2005}. 
	Used on a bitmap image, the algorithm finds which pairs of pixels need to be interchanged, to be able to store data hidden in the image. 
	This process is in many ways superior to methods such as LSB, due to the fact that very few pixels are actually changed while using this approach, 			pixels are 	merely interchanged.
	This makes it much more difficult to use statistical analysis on the image, to figure out if an image contains embedded data. 
	With JPEG images however, pixels are not set separately, and two pixels cannot be interchanged very easily. 
	Instead, this method can be used on the DCT coefficients, and have the data embedded into them.
	This approach will be described in-depth in section \ref{sec:graphtheory}.
\end{description}