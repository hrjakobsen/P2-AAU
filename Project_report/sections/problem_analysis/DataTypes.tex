\section{Image File formats}
After researching social media it has become apparent that images are a widely shared medium over social media and they seem less suspicious than, for example, linking to an external file. 
This section will study different image formats and how they can be used as both cover and message files.

Image file formats are a standardised way for storing and organising digital images.
Image files may store data in uncompressed, compressed or vector formats, though once rasterised the image becomes a grid of pixels each with a number of bits to designate its colour.

\subsection{Bitmap}
The BMP file format stores the colour components of each pixel separately, and therefore achieves a high file size. Though easy to work with, it is not a very common image file format to find on social media due to the file size.

The same goes for steganography, it is easy to hide information in the image, but the problems arise in that the file size of these images can become quite large in comparison to other more used image formats, and because of this, it is not used as widely as the the following image formats.

\subsection{Joint Photographic Experts Group}
JPEG is a widely used image format, which stems from the compression method used, as it is able to reduce the file size of an image substantially.
This compression has a drawback though, in that it is a lossy compression method, meaning a compression will remove data from the original image and therefore degrade the image quality.
Because the JPEG compression method is lossy, we cannot use simple methods like LSB for JPEG steganography because we do not have a good control over each individual pixel.

\subsection{Portable Network Graphics}
PNG offers a wide range of colour depths, from 24-bit (8 bits per channel) to 48-bit (16 bits per channel) and even up to 64-bit when an alpha channel is added.
The compression method used for the PNG image format is lossless, so no data is lost under compression.
Because the compression is lossless and PNG is widely used on the internet makes it ideal as a cover for use in steganography.

\subsection*{Conclusion}
After looking at the various file formats and how image compression works we have decided to work with the JPEG format.
As discovered in \ref{sec:whycompression}, we realised that most sites convert images to JPEG when uploaded. 
Because of this, JPEG appears to be the most commonly used file format when it comes to sharing images online.
This is why the JPEG format will be the focus throughout the rest of the report.
