\section{Why use Image Compression?}
Social media like Facebook, Twitter, and Imgur all use some form of image compression to reduce the size of user-uploaded images.
This is partly shown in table \ref{tab:PNG-compression}.
But why do these sites compress images?
An obvious reason is to make room for more data on the sites' servers.
This benefits the sites themselves.
Another reason is that not every social media user has an unlimited amount of internet data, which means that there is a certain pressure on the websites to be lightweight for the benefit of their users.
South Africa is a good example. According to a report from 2012, \citep{chetty_2012}, a vast majority of home-owners with internet access in South Africa have caps on their data plans.
Another example is the result of a poll by CivicScience \citep{tmobilemusic} cited by the Billboard magazine.
The poll suggested that 37\% of respondents avoid streaming music while on their phones due to fear of overages for using more data than what their plan allows.

Saving data for the users' benefit is therefore important for a lot of websites, perhaps even more so for social media sites, which often place focus on user-uploaded pictures.
Mobile users can also benefit even more from well-compressed images, since their data plans may not only be limited, but also slow.

A paper by Facebook from 2010 \citep{beaver2010} states that - at the time of the paper's writing - there were approximately 260 billion images on Facebook's servers, translating to over $2\cdot10^{16}$ bytes (20 petabytes) of data.
This means that in 2010, each image stored on Facebook's server was about 77 kilobytes in size after compression.
Looking at the test results shown in table \ref{tab:PNG-compression}, we can see that the uncompressed version of the images would have been close to 2 MB in size.
If Facebook should have stored the same images uncompressed, it would have required about $5.2\cdot10^{17}$ bytes (520 petabytes) instead.
And this is from 2010.

A 2014 edition of an annual report by the venture capital firm KPCB, \citep{meeker2014internet}, states that Facebook users alone uploaded about
350 million images per day to the website at the time of the report's writing.
The report shows that users uploaded about 200 million images per day in 2010.
Extrapolating from this data gives a net result of about 810 billion pictures on Facebook at the time of writing.
If the average size of a compressed image on Facebook is unchanged since 2010, the space required is about $6.2\cdot10^{16}$ bytes (62 petabytes).
Using the data from table \ref{tab:PNG-compression}, this would require about $1.6\cdot10^{18}$ bytes (1.6 exabytes) of space.

The calculations described in this section are just for Facebook and does not include backups. Instagram, Imgur,
and Snapchat are all based on user-uploaded images, and they all need to store a vast amount of information.
This is one of the reasons why image compression is so widely used on social media.
