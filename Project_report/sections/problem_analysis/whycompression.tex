\section{Why use image compression?}
Social media sites like Facebook, Twitter, and Imgur all use some form of image compression to reduce the size of user-uploaded images.
This is partly shown in table \ref{tab:PNG-compression}.
But why do these sites compress images?
An obvious reason is to make room for more data on the sites' servers.
This benefits the sites themselves.
Another reason is that not every social media user has an unlimited amount of internet data, which means that there is a certain pressure on the websites to be lightweight for the benefit of their users.
In fact, according to a report from 2012, \citep{chetty_2012}, a vast majority of home-owners with internet access in South Africa have caps on their data plans.
Unlimited data is fairly recent in South Africa as well, since all home-owners were capped prior to February 2010, according to the same report.
Saving data for the users' benefit is therefore important for a lot of websites, perhaps even more so for social media sites, which often place focus on user-uploaded pictures.\improvement{For the sake of mobile users as well}

A paper by Facebook from 2010 \citep{beaver2010} states that - at the time of the paper's writing - there were approximately 260 billion images on Facebook's servers, translating to over 20 PB of data ($1 PB = 1^{10} MB$).
This means that in 2010, each image stored on Facebook's server was about 77 kB in size after compression.
Looking at the test results shown in table \ref{tab:PNG-compression}, we can see that the uncompressed version of the images would have been close to 2 MB in size.
If Facebook should have stored the images uncompressed, it would have required about 52 PB instead.
And this is from 2010.

A 2014 edition of an annual report by the venture capital Firm KPCB, \citep{meeker2014internet}, states that Facebook users alone uploaded about
350 million images per day to the website at the time of the report's writing.
The report shows that users uploaded about 200 million images per day in 2010.
A back-of-the-envelope calculation hints at the total number of images on Facebook at the time of this report:
The average number of images uploaded to Facebook each day since 2010 multiplied by the number of days since 2010 until the date of this report's writing.
That is added to the number of photos already on Facebook in 2010.
This gives a net result of 810 billion pictures on Facebook today.
If the average size of a compressed image on Facebook is unchanged since 2010, the space required is about 62.4 PB.
Using the data from table \ref{tab:PNG-compression}, this would require about 156 PB of space.

The calculations described in this section is just for Facebook alone and does not include backups. Instagram, Imgur, 
and Snapchat are all based on user-uploaded images, and they all need to store an incredible amount of information.
This is one of the reasons why image compression is so widely used on social media.
