% -*- root: ../../DAT2-A423_Project_Report.tex -*-
\section{Magic Mike}
We have confined our problem to only images. After some discussion, we decided that images are the most shared format over social networks and seem least suspicious, as images are shared relatively frequently. After looking at the different image formats, we will work closely with the image format, JPEG, especially after our work in image compression. 

During our problem analysis, we initially thought that we would work with social media and how to share images containing hidden messages through them. We soon found out that each social media we tested had a different compression system. We came to the conclusion that if we worked with social media it was completely possible that trying to send images through their sites without being destroyed by their different compression systems was not a possibility. The problem that arises with social media is that we have no control over when and how they could choose to change their systems, thereby rendering our work useless.

We have also decided to work with a graph-theoretical approach as described in section \ref{sec:graphtheory} to work with the JPEG file format. This descision also allows us to work closely with the cluster-groups, where the mathematicians are also working with this approach. Having this shared method will help with our teamwork and should ultimately help both groups with understanding the method better and thereby forming a greater product.