\section{Introduction to Steganography}
Through the ages, people have been dependent on efficient forms of communication. In some cases, these messages would include information that should ideally be confidential and not read by anyone else other than the intended recipient. To do this, would require a way of concealing the details of the message in a way that an outsider would not be able to decipher what the actual message is.

Whereas cryptography is about concealing a message's content in a way that reveals that a message is concealed(which would undoubtedly raise some form of suspicion), steganography is about concealing the message's existence. This means that instead of the private message's content being protected by an obvious security-measure, people have had to find ways of circumventing interception and hide their messages in plain view to avoid any suspicion. This is one advantage steganography has over cryptography, since steganography can go unnoticed, while cryptography typically can't. 

Steganography is the art of concealing a message in a cover object, without having other people being aware that the concealed message is being sent \citep{Anderson1998}. The cover object could be in the form of video, audio or plain text. An example of this could be, hiding information in something innocuous that is not likely to get any unwanted, extra attention. The concealment should be subtle, so that unless a person was aware there was something to be found, is very unlikely to notice that a message is being passed in front of them without their notice.

As not everyone wishes to have every detail of their life, scrutinised and under the watch of everyone, people have had to implement forms of steganography to get their meaning across to their intended recipient. This goes back centuries, for example, people in ancient China would write messages on fine silk, which would then be pressed together until it was a much smaller piece of material and covered in wax. The cover of this message, was a person who would have to swallow the message to bring the information safely, without interception to the intended receiver \citep{Singh2001}. Another example being medieval Europe, where they had a system using templates, which would then be placed over a text, highlighting what the actual message is \citep{Anderson1998}.

Since then, forms of steganography have become much more advanced, and naturally, digitalised. Steganography can be implemented by anyone, no matter their intention, but would most typically be people who feel like they have something to hide. With the rise of social media, sending messages via the internet, and specifically these social media networks, has become the norm. Therefore it is only natural that a modern form of steganography should be able to work on these platforms. 
