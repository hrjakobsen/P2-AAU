\section{Ethical Questions and Steganography in Context}
As previously described, steganography is a practice with several different uses.
In this section we will study different uses digital steganography.\vspace*{12pt}

\noindent The need to hide a message has historically been shown to be beneficial in times of war and revolution \citep{Singh2001}.
Steganography has been seen used by people in war, by terrorists, activists, dissidents and suppressed citizens as a means of secretly sending a message ideally without the existence of the message being known to anyone, but the sender and intended receiver.
For example these messages could contain war targets, locations of allies, military orders or simply express political opinions without endangering the sender of the message.

With all these different uses of steganography, good and bad, the question ``Is it really ethical to develop tools that makes this kind of secret communication possible?'' arises.
Keeping a realistic mindset, one should assume that a big percentage of the uses of steganography would involve criminal activity or terrorism.
On the other hand, steganography has plenty of uses, some of which will be covered in the next section.

\subsection{Uses of Steganography}
To further study steganography in a real-life context and the relevance to the aforementioned question, it is important to examine the uses of this practice, and to do this a few examples of digital steganography will be described.

A hypothetical example could be that a technology company is working on a new advanced product that they do not want anyone to know about before receiving a patent on all used production technologies or before a full product release.
If they share a file containing related information protected by some sort of encryption, it is obvious that they are trying to send an important message that they wish to hide from press and competition.
If they are able to send a message, concealed in a cover that seems innocent and perhaps unrelated to product development, the company will not draw attention to themselves in the same way.

Film companies today also make use of steganography to hide a message that is essentially a watermark.
In every copy sent out to cinemas and other services that show the film, a few pixels, will be slightly altered to give every release-service its own unique ID.
This means that the publisher will be able to determine what service their film was leaked from if that was to happen.
The difference in the picture will be so subtle that it is hardly visible to the human eye.

Music companies make use of the same principle but instead alter specific audio frequencies slightly.
This slight differences in the track will be inaudible to the human ear \citep{Anderson1998}.

These examples demonstrate that the uses of steganography may not be as black and white as they initially seem.
It really does hold a lot of possibilities and the majority of these can be easily justified.