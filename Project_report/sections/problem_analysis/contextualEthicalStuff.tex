\section{Steganography in a Context and Ethical Questions}
As previously described, steganography is a practice with several different uses.
Digital stego is not just a programmer's hobby and his or hers wish to show off, but has practical uses and must be viewed in the context in which it is used.
The need to hide a message has historically shown to be great in trouble- and fearsome times such as times of war and revolution.
Steganography has been seen used by people in war, by terrorists, activists and suppressed citizens as a means to secretly send a message without the mere existence of the message being known to ideally anyone else than the sender and intended receiver.
These messages could be containing for example war targets, locations of allies, military orders or simply express, perhaps political, opinions without endangering the sender of the message.


With all these different uses of steganography, good and bad, the question ''Is it really ethical to develop such tools that makes this secret communication possible?'' quickly arises. Keeping a realistic mindset, one could very likely assume that a big percentage of the uses of steganography would be involving criminal activities and terrorism. Would you really have anything to hide unless you have evil intentions?

\subsection{Examples on uses of steganography}
To further investigate steganography in a real-life context and the relevance to the aforementioned question, it is important to examine the uses of this practice and to do this, a few examples will be described.

A thought up example could be that a technology company is working on a new advanced product that they do not want anyone to know about before receiving patent on all used production technologies or before a full product release.
If they share a file containing related information protected by some sort of encryption, it is obvious that they are trying to send a message that is important to the company and that they have something that they wish to keep from press and competition.
If they are able to send a concealed message, hidden in plain sight inside something that seems completely irrelevant to product development, the competition will never come to have great interest in the company and it's procedures.

Also film companies today use steganography to hide a 'message' that is essentially a watermark. In every copy send out to cinemas and other services that display a film released by the company, a few pixels, different from service to service, will be slightly altered to give every copy its own 'ID'.
This means that the producer would be able to determine by what service their film was leaked if a leak was to happen.
The difference in the picture will be so subtle that it will not be visible to the human eye.

Music companies make use of the same principle but instead alters specific frequencies slightly. This slight difference in the track will be inaudible to the human ear. \cite{Anderson1998}


These examples demonstrates that the uses of steganography may not be as black and white as it initially seems. It really does hold a lot of possibilities and most of these can easily be justified.

