% -*- root: ../../DAT2-A423_Project_Report.tex -*-
\section{Implementation of a JPEG image}

\subsection{Working with bits}
While working with JPEG images, bit-patterns and bit-wise operations often occur.
Due to hardware limitations, a single bit in the memory cannot be changed individually, and we can only address the individual bytes.
There are now two ways that we can go about this:
Either we use bytes to represent bits, and effectively waste 7 bits of memory for each bit, or we use the class \lstinline|BitArray| from the \lstinline|System.Collections| library. What this class does, is that it handles the packing of the bits into bytes, and makes you able to adress the individual bits.

The \lstinline|BitArray| seemed like the way to go, and we implemented a class \lstinline|BitList| which implemented List-like features like \lstinline|Add| and \lstinline|Insert| while using the \lstinline|BitArray| to store our data.
The implementation can be seen in appendix \ref{app:C}. 

The \lstinline|BitArray| offers the functionality of modifying the length of the array, so that more values can be added.
This means that every time the \lstinline|BitList| runs out of space in the underlying \lstinline|BitArray|, we simply multiply the length of the array by 2, so that more values can be added.
This makes appending to the array relatively fast, since we have enough room to add more values most of the time.
Inserting values in the middle of the array proved to be much more difficult, however.

When inserting a value into the array, all values after the value to be inserted have to be shifted, so that room is made from the new value.
In an example 512x512 image, we have over 750,000 bits to save, and if we were to insert a bit in the beginning of the array, we would have to move 750,000 elements in the array.
All in all a very computationally expensive operation.

So while the \lstinline|BitArray| seemed promising in theory, saving us a lot of memory, the need of inserting bits into the middle of the array makes using the \lstinline|BitArray| infeasible.
What we need to solve this problem is to eliminate the need to shift a large part of the array. 
One way of solving this problem is to make a data type which combines the functionality of the \lstinline|BitArray| with a linked list. 
By splitting up the large \lstinline|BitArray| into smaller linked arrays we could shift the values in the smaller arrays. 
We would have a slightly longer access time, but have a much faster insertion time. 
Another way to solve the problem, and what we ended up doing was to eliminate the need of inserting the values. 
We realised that the only time we need to insert values into the \lstinline|BitList| was when there were 8 consecutive ones in the scan data. 
So what we did what created the method \lstinline|CheckedAdd| which we would use while writing scan data to the list. 
This method would then keep track of the last inserted values, and insert the zeroes on-the-fly instead of after the whole process. 
The method \lstinline|CheckedAdd| can also be seen in appendix \ref{app:C}