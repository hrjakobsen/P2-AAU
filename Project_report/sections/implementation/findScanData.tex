\subsection{findScanData}
The function \lstinline|findScanData()|, reads through the input until it meets the marker 0xFFDA, then skips the next 12 bytes which contains information about the scandata section.
Once this is done, the function will add every single byte in the scandata section to a \lstinline|List<byte>| list until a marker is found.
Since the decoding-process works on a bit-level, we'll need to convert this list of bytes into an array of bits instead. To do this, we've got the \lstinline|BitList| class, which suits our needs.
To do this, we apply eight different masks to each byte to get the value for each bit, then add a true or false value to a new list of bits. 
This list, will be the list returned by \lstinline|findScanData()|
