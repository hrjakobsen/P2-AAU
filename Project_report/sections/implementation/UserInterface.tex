% -*- root: ../../DAT2-A423_Project_Report.tex -*-
\section{Implementation of Custom Tables in the GUI}
In section \ref{sec:designUI} the need for custom options was described. One of these options was the possibility to choose custom Huffman and quantization tables. This is implemented into our programme as a custom component for the Forms designer.

We want the user to be able to define their own Huffman tables. This is done using textboxes, which will be converted into entries of a \lstinline|HuffmanTable|. We want these textboxes to function as a joint set so that we can easily move the textboxes around and have them use them as one component.

To do this, a class \lstinline|HuffmanTableComponent| inheriting from the \lstinline|System.Windows.Forms.Panel| class was created. 

This class, taking a \lstinline|HuffmanTable| in its constructor, creates a \lstinline|Panel| and then for each \lstinline|HuffmanElement| in the given \lstinline|HuffmanTable| adds a \lstinline|TextBox| to the \lstinline|List<TextBox>| \lstinline|runSizeBoxes| and sets its size, position, max length and font and then a \lstinline|TextBox| to the \lstinline|List<TextBox>| \lstinline|codeWordsBoxes| and sets the same properties. This can be seen in listing \ref{HuffmanTableComponent}.

\begin{lstlisting}[firstnumber=18,label=HuffmanTableComponent,caption={\lstinline|HuffmanTableComponent| constructor \textbf{File: }HuffmanTableComponent.cs}]
public HuffmanTableComponent(HuffmanTable huffmanTable) {
  Table = huffmanTable;
  var elementList = Table.Elements.ToList();
  Size = new Size(410, 244);
  
  for (int i = 0; i < Table.Elements.Count; i++) {
    addCodeWordsBox(i);
    
    string codeWord = Convert.ToString(elementList[i].Value.CodeWord, 2);
    
    if (codeWord.Length != elementList[i].Value.Length) {
      codeWord = codeWord.PadLeft(elementList[i].Value.Length, '0');
    }
    
    codeWordsBoxes[i].Text = codeWord;
    
    addRunSizeBox(i);
    runSizeBoxes[i].Text = Convert.ToString(elementList[i].Value.RunSize, 0x10).PadLeft(2,'0');
  }
  
  InitializeComponent(); 
}
\end{lstlisting}

The class also implements \lstinline|AddRow()| which adds one more row as seen in listing \ref{AddRow}.

\begin{lstlisting}[firstnumber=80,label=AddRow, caption={\lstinline|AddRow()| method \textbf{File: }HuffmanTableComponent.cs}]
public void AddRow() {
  int j = codeWordsBoxes.Count();
  
  codeWordsBoxes[0].Select();
  VerticalScroll.Value = 0;
  
  addCodeWordsBox(j);
  addRunSizeBox(j);

  codeWordsBoxes[runSizeBoxes.Count() - 1].Select();
}
\end{lstlisting}

Lastly a \lstinline|SaveTable()| method is implemented as seen in listing \ref{HuffmanSaveTable}.
This method begins with the initialisation of a new \lstinline|HuffmanTable| and then uses a for-loop to run through the number of rows in the component.
This for-loop checks if each pair of text boxes are empty.
If they are, we skip the text boxes.
If they are not, the entries in the text boxes are added as the runSize and the codeword of a  \lstinline|HuffmanElement|. 

\begin{lstlisting}[firstnumber=137,label=HuffmanSaveTable, caption={Save method of the \lstinline|HuffmanTable| method \textbf{File: }HuffmanTableComponent.cs}]
public HuffmanTable SaveTable() {
    HuffmanTable h = new HuffmanTable();
    
    for (int i = 0; i < codeWordsBoxes.Count; i++) {
      if (string.IsNullOrWhiteSpace(runSizeBoxes[i].Text) || string.IsNullOrWhiteSpace(codeWordsBoxes[i].Text)) {
        continue;
      }
      
      byte runSize = Convert.ToByte(runSizeBoxes[i].Text, 16);
      ushort codeword = Convert.ToUInt16(codeWordsBoxes[i].Text, 2);
      h.Elements.Add(runSize, new HuffmanElement(runSize, codeword, (byte)codeWordsBoxes[i].Text.Length));
    }
    return h;
  }
}
\end{lstlisting}

A custom component \lstinline|QuantizationTableComponent| is implemented in an almost identical way, only simpler, as a \lstinline|QuantizationTable| is always the same size, 64.
The final version of the graphical user interface can be found in appendix \ref{app:C}.