\section{Decoding}
As previously mentioned in section \ref{sec:DesignHuffman}, a stream-reader reads from a file from the beginning towards the end, thus the first thing we did in the decoding process, was to read the huffman tables from the JPEG file. 
To do that, we implemented algorithm \ref{alg2}, which gave us all the information we needed about the Huffman table, and what kind it was so we could store it in the appropriate property. 
Checking the first four bits in the output parameter, ensured we knew whenever the Huffman table was a DC table, or an AC table. 
Reading the remaining four bits gave us the ID of it, which then told us whenever the table was for the luminance channel, or the chroma components.
After this, it is now time to read in all the bytes from the scan-data section. So long no header is being read, the byte will be written to a byte array, which will then be turned into a bitarray.
Working on the bitarray is the fastest possible solution, since we'll read each bit from start to end and won't have to add additional bits in the future.
