\section{Decoding}
As mentioned in section \ref{sec:DesignHuffman}, a \lstinline|StreamReader| reads from a file from the beginning towards the end.
Thus, the first thing we did in the decoding process was to read the Huffman tables from the JPEG file. 
To do that, we implemented algorithm \ref{alg2}, which gave us all the information we needed about the Huffman table, including what kind it was, so that we could store it in the appropriate property. 
Checking the first four bits in the output parameter ensured that we knew whenever the Huffman table was a DC table, or an AC table. 
Reading the remaining four bits gave us the ID of the Huffman table, which told us whenever the table was for the luminance channel, or the chroma components.
After this, it is now time to read in all the bytes from the scan-data section. As long as no header is being read, the byte will be written to a byte-array, which will then be turned into a \lstinline|BitArray|.
Working on the \lstinline|BitArray| is the fastest possible solution, since we will read each bit from start to end and will not have to add additional bits in the future.
