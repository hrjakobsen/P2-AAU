\section{Decoding}
As mentioned in section \ref{sec:DesignHuffman}, a \lstinline|StreamReader| reads from a file from the beginning towards the end.
Thus, the first thing we did in the decoding process was to read the Huffman tables from the JPEG file. 
To do that, we implemented algorithm \ref{huffmanAlgo}, which gave us all the information we needed about the Huffman table, including what kind it was, so that we could store it in the appropriate property. 
Checking the first four bits in the output parameter ensured that we knew whenever the Huffman table was a DC table, or an AC table. 
Reading the remaining four bits gave us the ID of the Huffman table, which told us when the table was for the luminance channel, or the chroma components.
After this, it was then time to read in all the bytes from the scan-data section. As long as no header was being read, the byte would be written to a byte-array, which would then be turned into a \lstinline|BitArray|.
Working on the \lstinline|BitArray| is the fastest possible solution, since we would read each bit from start to end and would not have to add additional bits in the future. The \lstinline|BitArray| implementation however, lists each bit in reverse order, meaning we would have to read from the array in reverse as well, which would likely affect performance considering how caches work.
To avoid this, we would instead have to use the \lstinline|BitList| class again, which would add each bit in the correct order, and since a \lstinline|List| is just an array which size can be dynamically increased as required. 

The Decoder implements an interface, which specifies the decoder to have four properties, each containing the respective Huffman table, and the method \lstinline|decode()|, which returns a \lstinline|byte| array. 
The Huffman tables are read from a stream from a given filepath. This stream gets assigned to a \lstinline|BinaryReader|, since what we would read, would be bytes. All of this can be seen in the constructor in Listing \ref{code:decoderConstructor}.
\lstinputlisting[firstline=36, firstnumber=36, lastline=54, label={code:decoderConstructor}, caption={The Decoder constructor}]{../Programmer/Stegosaurus/Stegosaurus/JPEG/Decoder.cs} 

The listing also shows how we find the HuffmanTable and store it in the right property, based on its ID and Class.

Once we have read in all the lengths of the Huffman-encoded values, the next we will be reading, is the runsize of the Huffman codes. To get the correct code, we will bitshift the current value we have by one to the left, once there are no more huffmancodes to read of the current length, then increment the current length by 1, as described in algorithm \ref{huffmanAlgo}.

After implementing the rest of the algorithm, we will end up with a complete \lstinline|HuffmanTable|.

We do this four times, since we there are four Huffman tables.

If the public method called \lstinline|decode()| is called, the method will call appropriate methods and sub-methods to decode the message hidden in the jpeg image with the Graph-Theoretic-Approach to Steganography.
The method \lstinline|findScanData()| returns a bitlist, which will be used by the method \lstinline|decodeScanData()|, and then return the message to the user who called the method.
\subsection{findScanData}
The method \lstinline|findScanData()|, reads through the input until it meets the marker 0xFFDA, then skips the next 12 bytes which contains information about the scandata section.
Once this is done, the method will add every single byte in the scandata section to a \lstinline|List<byte>| list until a marker is found. The implemented method can be seen in Listing \ref{code:findScanData}
\lstinputlisting[firstline=110, firstnumber=110, lastline=141, label={code:findScanData}, caption={Read in the ScanData section into a \lstinline|BitList|}]{../Programmer/Stegosaurus/Stegosaurus/JPEG/Decoder.cs}
Since the decoding-process works on a bit-level, we will need to convert this list of bytes into an array of bits instead. To do this, we have to get the \lstinline|BitList| class, which suits our needs. This is shown in lines 134-139.

This list, will be the list returned by \lstinline|findScanData()|.

\subsection{decodeScanData}
The function \lstinline|decodeScanData(BitList)| takes a \lstinline|BitList| as input, and returns the encoded message.
The function got multiple subfunction, each assisting the \lstinline|decodeScanData| function in finding the message. The functions are split up as follows, with their own functionality.
\begin{itemize}
	\item \lstinline|_addNextMCU(List<int>, BitList, ref int)| which finds four Y-component blocks, and two chrominance blocks. Finding the blocks is taken care of by the \lstinline|getBlock| function. The values returned by \lstinline|getBlock|, will be added to \lstinline|List<int>|, which it received as input.
	\item \lstinline|getBlock(BitList, ref int, HuffmanTable, HuffmanTable)| which iterate through the scandata saved in the \lstinline|BitList|, and for each bit it has read, will look up the first value in the AC Huffman table, and will look up the remaining 63 values in the AC table, or until the EOB value is found.
	These values will be saved in an \lstinline|int[]| in zigzag ordering.
	Once all 64 values or EOB has been found, all values that are not 0, will be added to a \lstinline|List<int>|, then returned to the caller.
\end{itemize}
Since the length of the message, and what modulo is needed to decode the message, is hidden within the first 16 values, the function \lstinline|_addNextMCU|, will have to be called until we at least have 16 values.
Once this is done, we can find the length of the message, modulo and can thus start reading in the remaining values by calling \lstinline|_addNextMCU| the appropriate amount of times.
To get the length and modulo, the following is done
\begin{enumerate}
	\item Call \lstinline|getLength(List<int>)|, which finds the length of the message by adding the first 14 values together in pairs of two, then doing modulo 4 of that value. 
	This value will be added to a \lstinline|ushort| value, which will first be bitshifted left twice. 
	After running this loop seven times, the \lstinline|ushort|, will be the length of the encoded message.
	\item Call \lstinline|getModulo(List<int>)|, which will add the 14'th and 15'th value together, then after doing modulo 4 on the result, look through a switch with 3 values in it, then return the right modulo based on the result of the operation.
\end{enumerate}
With the values from these two functions, it's now just a matter of calling \lstinline|_addNextMCU| until the values containing the message, has been found.
When all the required values has been read and saved in the \lstinline|List<int>| variable, the message can be found by doing the following
\begin{enumerate}
	\item add all the values in pairs of two together, then do the modulo operation on them using the value of modulo found with the \lstinline|getModulo| function.
	\item iterate through the message bits, adding them together to a single value, then shift the value to the left depending on the value of modulo.
	This will be done until the value can no longer be stored in a single byte, then added to a \lstinline|List<byte>| variable, and continue from where we left off until there's no longer any message bits left.
	\item The \lstinline|List<byte>| variable now contains the decoded message and will be returned to the caller of \lstinline|decodeScanData()|, as a \lstinline|byte[]|
\end{enumerate}

The value of the \lstinline|decodeScanData()| method, will be returned to the caller of the \lstinline|decode()| method.
