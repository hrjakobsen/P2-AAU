\section{Cluster group}
%During the project we have worked with two groups from different studies from Aalborg University, one from Mathematics, another from Internet Technology and Computer Systems (ITC). 
%The maths team researched different algorithms related to graph theory and steganography, while the ITC group was required to implement a system allowing two devices to communicate without having direct access to each other.
%The ITC group could use our product through an API in their communication protocol, and the maths team supplied us with insight into the algorithms they researched, which allowed us to implement more advanced versions of them, ultimately helping both groups to understand the algorithm better.
%In the end, all of our modules are able to stand as individual products, but when combined form something greater.
This section focuses on the cluster project, which was supposed to be a large part of the process. The section will describe the various concerns that our group had before and during the project, as well as an analysis of the process that we went through. Finally, a paragraph will conclude on the outcome of the cluster project, and what might be improved.

\subsection{Motive}
Before the groups for this project were formed, everyone was encouraged to read a project catalogue containing all the project proposals available for the project.
The steganography entry in the catalogue contained a paragraph that read (translated from Danish):
\begin{quote}\textit{This project proposal is provided to first-year students from three different subjects: Mathematics, Computer Science/Software Engineering, and Internet Technologies and Computer Systems.
The groups who chose the proposal are required to make their own project based on their respective subjects, but the end goal is to coordinate the projects, so that each one of them are a part of an overall project.
Doing it this way, this project will resemble the challenges that you will face after graduation.}\end{quote}
%Include source?

\subsection{Concerns}
There have been a number of cluster meetings with the groups from Mathematics (MAT) and Internet Technologies and Computer Systems (ITC), and us, Computer Science (DAT). as well as with each group's project advisors.
In the beginning there was a bit of uncertainty in our group about how the entire process would work, since each group in the cluster had a different curriculum and a different set of requirements to be fulfilled.
Futhermore, we were concerned about the fact that each group had to figure out a part of the whole solution based on the work of others.
Essentially, it sounded like MAT would develop an algorithm as a base for us (the Computer Science group) to use.
DAT would then use this algorithm and develop a class library for the ITC group to use.
ITC would finally use the algorithm and the library in their project.
This would require a lot of time and be largely inefficient, since one step of the process had to be completed before another one could begin, and we only had about four months for the project in total.
It would also mean that ITC would be dependant on DAT, while DAT would be dependant on MAT for their respective projects, and that would not be feasible.
In a similar vein, the point of the cluster group was that each group could benefit from the work of others.
But how would our group benefit from the work of ITC? And would MAT gain any benefit from the cluster group at all?

We brought these concerns to the table at the first cluster meeting with each group's advisor and Hans Hüttel, who made the project proposal.

\subsection{Trying to work together}
There was no real progress in the cluster collaboration during the first weeks, which was expected.
Everyone was busy learning about steganography and how to use it.
Our group made a fairly trivial product to embed data into other data using least significant bits, but we could not spend the entire project on that.
ITC could use our product though, and it was decided that DAT would deliver either that or whichever better product that was developed later.

Olav Geil, MAT's advisor, held a lecture on an article about a graph-theoretic approach to steganography.
This proved valuable to both us and MAT, as both groups would later decide to use this approach in some form.
DAT and MAT somewhat connected over this, thinking this could be the key to the lacking collaboration.
We were encouraged by the advisors to collaborate, perhaps letting MAT develop and giving us an efficient algorithm using this approach, while DAT developed a slightly less efficient algorithm for testing purposes until MAT was done.
When voicing concern about how MAT would benefit from this, we were told that it could possibly be exciting for them to see their algorithm in action.

To solidify the deimise of the cluster project, we realised on a cluster meeting - a month before deadline - that DAT and MAT had chosen different paths.
MAT had dedicated their time to make the algorithm presented by Olav Geil as precise as possible, sacrificing efficiency, while DAT were looking for an efficient algorithm where extreme precision had less of a priority.
This meant that MAT and DAT would be of no use to each other, and it was decided that it was the end of the collaboration between DAT and MAT.
ITC will still get a complete steganographic solution to use as a library with select methods exposed, since they will have no use of the entire codebase in their project.

\subsection{Lessons learned}
It is hard for three groups with completely different curriculums, ideas, and requirements to come together and work on a shared project.
Given enough time and more in-depth talks with each other, it could most likely work, but neither group was prepared enough for the challenge.
It seems like this kind of cluster project is well-suited for a corporate setting, where the goal is to develop a complete solution across all teams.
In a university setting, however, each group is required to learn and possibly develop something based on their respective curriculum, and it makes sense that they first and foremost focus on that with their own project, and not on cluster work, which can only be considered a ``bonus objective'' that can be skipped.

