% -*- root: ../../DAT2-A423_Project_Report.tex -*-
\section{Testing private methods}
The goal of unit tests is to make sure that the code works as intended.
If the unit tests are thorough, they become a way to prove that the class does everything in the specifications, and shows that other people can count on the class as well.
Now if someone want to use a class, they should only have access to the public methods and properties, this means that as long as the unit tests shows that the public methods and properties of a class works, they do not have to explicitly show that the private methods works as intended.

Of course, if the private methods does not work, there is a big chance of the public methods not working either.
But as a developer we make no guarantees that the private methods have any effect, as only the public methods are what we have chosen to let other people use. 
So to test the private methods, one should test the public methods thoroughly enough, that every part of the private methods are tested as well. 

So while data-hiding makes it much easier to implement other peoples work, and makes it much more clear what a certain class offers of opportunities, we lose some of the flexibility when testing our code, as unit testing is basically an implementation of the class under test, and seeing if that class gives the expected result, given certain criteria. 

Our program exists of multiple classes, but a lot of the work is done in private methods in the class \lstinline|JPEGImage|, and the only way we would be able to test the output is to check the file which the public method \lstinline|Save(string)| can provide.

Testing every logical statement in the program with the aforementioned method, would certainly break one of the principles of unit tests, that they should be quick to run.
We want it so that with the press of a button, we can quickly know if something has broken due to a change somewhere in the program, so having to wait multiple minutes on images being created in full and tested byte-for-byte, to know if we broke something in one private method, would result in the unit tests not being run as often. 

Of course people before us has run into this problem, and solutions are available.
Microsoft offers a library called \lstinline|Microsoft.Something.About.Testing| which contain the class \lstinline|PrivateObject| and \lstinline|PrivateType|.
With \lstinline|PrivateObject| we can pry open an object, and access private methods on an instance via reflection. \lstinline|PrivateType| similarly allows us to access private static members. 

The syntax becomes somewhat awkward as we have to rely on strings containing method names to access the private members, but it does offers possibilities to tests the private method much more easily that through the public methods as described earlier.