% -*- root: ../../DAT2-A423_Project_Report.tex -*-
\section{Decoding}
As an encoded message is of no use if it cannot be retrieved, this section will describe in detail how the decoder is created.

\subsection{Find the Huffman Tables from a JPEG file}
\label{sec:DesignHuffman}
To find the encoded text using the Graph-Theoretic Approach, the first step is to read in the Huffman tables from the JPEG-file, making it possible to look up the code written in the scan-data session of the file.
As mentioned in section \ref{sec:jpegStudy}, the Huffman tables are placed in the beginning of a JPEG-file, prior to the scan-data session.
Since a \lstinline|StreamReader| reads from the beginning of a file towards the end, it is important to proceed in much the same way.
The first step is to look for the $DHT$ marker $0xFFC4$, since we then know that what follows is all the information about the Huffman table.
To read in all the information we need, and store it in a table, we came up with algorithm \ref{huffmanAlgo} with a little inspiration from \cite{HuffmanDecoding}
\begin{algorithm}
\caption{Find Huffman table in a jpeg file}
\label{alg2}
\begin{algorithmic}
\REQUIRE Binary Reader, which is reading a JPEG file, and an output parameter so the caller knows the table's ID and Class
\ENSURE A Huffman Table fully constructed from the bytes in the JPEG file

\STATE{$insideHuffmanTable := false$}
\WHILE {$insideHuffmanTable = false$}
	\STATE {$marker =$ read byte from stream}
	\IF {$marker = 0xFF$}
		\STATE {$marker = $ read byte from stream}
		\IF {$makrer = 0xC4$}
			\STATE{$insideHuffmanTable := true$}
		\ENDIF
	\ENDIF
\ENDWHILE
\STATE{$length := $read byte from stream $<< 8 + $read byte from stream $- 19$}
\STATE{$ClassAndID := $read byte from stream}
\FOR{$i:=1$ \TO $i=16$}
	\STATE{array of lengths spot i $ := $read byte from stream}
\ENDFOR
	\STATE{elements of current length $:=$ array of length spot 1}
	\STATE{current code representation $:= 0$}
	\STATE{current code length $:= 0$}
	\FOR {$i:=1$ \TO $i = length$}
		\WHILE{elements of current length $= 0$ \AND current code length $< 17$}
			\STATE{current code length $+ 1$}
			\STATE{current code representation $<< 1 $}
			\STATE{elements of current length $:= $ array of lengths spot current code length}
			
		\ENDWHILE
		\STATE{Create new Huffman element from the next read byte in the stream, with the same code as the current code representation and with the length from the array of lengths from the spot current code length}
		\STATE{Add Huffman element to table}
		\STATE{elements of current length $- 1 $ }
		\STATE{current code representation $+ 1$}
	\ENDFOR
\RETURN $Huffman Table$
\end{algorithmic}
\end{algorithm}



\subsection{Decoding the Huffman-encoded Values}
As explained in section \ref{sec:graphJPEG}, the hidden message will be hidden in pairs of two quantized values using the $\oplus_m$ operator.
To get the data out, we have to read the first 16 non-zero AC values, in order to find what $m$-value the message is hidden with, and the length of the message.
We can then use this length and $m$-value to determine how many non-zero AC values to read, and decode the original message.
Finding the length of the message, as well as the value of $m$, requires you to decode the scan data as follows:
\begin{enumerate}
	\item Read 1 Huffman DC element for the Y channel
	\item Read 63 Huffman AC elements for the Y channel, or until EOB
	\item Repeat steps 1 and 2, three more times
	\item Read 1 Huffman DC element for the Cb channel
	\item Read 63 Huffman AC elements for the Cb channel, or until EOB
	\item Read 1 Huffman DC element for the Cr channel
	\item Read 63 Huffman AC elements for the Cr channel, or until EOB
	\item Repeat steps 1 through 7 until 16 non-zero AC values are found
\end{enumerate}
Reading the Huffman-encoded values in accordance to the table shown in section \ref{tab:huffmanCodes} is not completely trivial.
The values with $1_2$ as their most significant bit, are positive numbers, and can be interpreted as an unsigned data type.

To get the negative values, we have to do some calculations. 
First, we notice how with each category the minimum number is the same as adding one to the $2^n$ negative value, and the higher the bit-value becomes, the higher the actual value becomes as well. 
Based on this information, we have derived algorithm \ref{algDecodeHuffmanValue} to calculate the number from a category and a bit-string.
\begin{algorithm}
\caption{Decode the huffman-encoded value}
\label{algDecodeHuffmanValue}
\begin{algorithmic}
\REQUIRE A bitstring $s$ of 16 bits or less, and a category $n$
\ENSURE A decoded huffman value

\STATE{$returnValue := 0$}
\IF {most-significant bit of $s = 1$}
	\RETURN $s$
\ENDIF
\STATE{$returnValue := s + 1 - 2^n$}
\RETURN $returnValue$
\end{algorithmic}
\end{algorithm}

These values, and the preceding zeroes, will have to be read into an array in zig-zag ordering.

\subsection{Decoding the Message}
We will need 16 non-zero values in order to find the length of the encoded message and the $m$-value, which we will need to use in order to decode the message itself.
To find the length from the first 16 values, we combine the first 14 values into pairs of two.
For each of those pairs, we then use the $\oplus_4$ operator, and combine the result from the 7 pairs into a bit-string with length 14 because each result can be represented by two bits. This bit-string can then be converted to a number which is the length of the message.

The $m$-value is hidden within the remaining values. We use the $\oplus_4$ operator again, and the result describes what $m$-value should be used for decoding the message itself.
There are four different combinations of those two bits.
Only three of them are valid: $00_2$ means that the $m$-value is $2$, $01_2$ means the $m$-value is $4$ and lastly $10_2$ means that it is $16$.

Once the length has been found, we can keep decoding the scan data segment until enough AC coefficients have been read to decode the full message, using the modulo value we just found.

When enough values have been read, the decoding process of the message is the same as listed in the beginning of this subsection, with the only difference being that the value of modulo may be different. 
Each byte will then have to be written to an array and returned to the user.
This array will then contain the message that was encoded in the JPEG image.