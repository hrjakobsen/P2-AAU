% -*- root: ../../DAT2-A423_Project_Report.tex -*-
\section{Design of the JPEG image encoder}
\label{sec:designJPEG}
In section \ref{sec:jpegStudy} we examined how a JPEG image is encoded and saved to a file.
In this section we will use this knowledge to design a JPEG encoder that can be implemented into our program.
By making the encoder, we hope that we can create a valid JFIF file where we have also implemented the graph theoretical approach as described in section \ref{sec:graphJPEG}.
As we can recall from section \ref{sec:jpegStudy}, the conversion from a bitmap image to a JPEG image is the result from the following process:

\begin{enumerate}
	\item The image is split into three channels (Y, Cb and Cr).
	\item The image is divided into MCU's where a down-sampling may take place.
	\item DCT is performed on all 8x8 blocks in the MCU's.
	\item Each new 8x8 block of DCT coefficients is quantized.
	\item All 8x8 quantized values are then Huffman-coded and saved to a file using zero-runlength encoding.
\end{enumerate}

Some of these processes such as DCT are simply mathematical formulas performed on the data, and will not change from one image to another.
However, other processes such as Huffman coding and quantization can yield very different results, depending on what Huffman- or quantization tables were used.
If a quantization table with low values is used, the quality of the image would be higher, as the quantization process will not make as many of the entries in the 8x8 block zeroes.
A downside of quantization values being low is that because of the low amount of zeroes in the 8x8 blocks, the zero-runlength encoding would not be as efficient, which would mean a higher file-size.

Because the Huffman- and quantization tables have such a tremendous impact on the resulting JPEG image, it would make sense to have the user able to change these to their liking.
Because of this, both quantization tables and the Huffman tables will be created as classes of their own, and because of this, changing the tables in the encoder will become as easy as changing a property.

Furthermore, a Huffman table is a collection of run-sizes and a bitstring of how to encode the run-size.
These bitstrings can be saved in an unsigned data type, but because these bitstrings can start with the bit 0, there is no way to tell the length from the unsigned data type alone.
Because of this, a class \lstinline|HuffmanElement| was created.
The class then contains the unsigned data type along with the length of the bitstring.

So what we have is a Huffman table where we want to lookup a byte (the runsize) and get the corresponding bit encoding of the byte. To do this, this class must be implemented using a dictionary, where you can use the runsize as the key, and get the corresponding bitstring back. That way we can very efficiently translate from bytes to bit-encoding in the file. 


To make the program easier to expand at a later time, the JPEG image class will implement the interface \lstinline|IImageEncoder|. Using an interface makes it much easier to later switch the implementation, or in our case the encoding method. If someone wanted to encode a PNG image instead of a JPEG image, they would just switch the \lstinline|JPEGImage| instance to for example a \lstinline|PNGImage|
Implementing the interface means that the \lstinline|JPEGImage| class will have the following public methods:

\begin{itemize}
	\item \lstinline|public void Save(string path)| which saves the image to a file at the given path.
	\item \lstinline|public void Encode(byte[] message)| which embeds a message into the image.
	\item \lstinline|public int GetCapacity()| which calculates how much information can be stored in the image.
\end{itemize}

Another reason for using the \lstinline|IImageEncoder| interface is that we can keep working on the actual implementation of the JPEG-encoder, while the ITC-group can build their system around the interface. The interface is very sparse, but offers the basic functionality which should be present in all types of image-encoders.