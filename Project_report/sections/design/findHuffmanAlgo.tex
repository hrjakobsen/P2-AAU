\begin{algorithm}
\caption{Find Huffman table in a jpeg file}
\label{huffmanAlgo}
\begin{algorithmic}
\REQUIRE bitstring, which is reading a JPEG file, and an output parameter so the caller knows the table's ID and Class
\ENSURE A Huffman Table fully constructed from the bytes in the JPEG file

\STATE{$insideHuffmanTable := false$}
\STATE{array of lengths $AoL$}
\STATE{elements of current length $e$}
\STATE{current code length $cL := 0$}

\WHILE {$insideHuffmanTable = false$}
	\STATE {$marker =$ read byte from stream}
	\IF {$marker = 0xFF$}
		\STATE {$marker =$ read byte from stream}
		\IF {$marker = 0xC4$}
			\STATE{$insideHuffmanTable := true$}
		\ENDIF
	\ENDIF
\ENDWHILE
\STATE{$length :=$ read byte from stream $<< 8 +$ read byte from stream $- 19$}
\STATE{$ClassAndID :=$ read byte from stream}
\FOR{$i:=1$ \TO $i=16$}
	\STATE{$AoL_i :=$ read byte from stream}
\ENDFOR
	\STATE{$e := AoL_1$}
	\STATE{$code := 0$}
	\STATE{$cL := 1$}
	\FOR {$i:=1$ \TO $i = length$}
		\WHILE{$e = 0$ \AND $cL < 16$}
			\STATE{$code * 2$}
			\STATE{$cL + 1$}
			\STATE{$e :=  AoL_{cL}$}
		\ENDWHILE
		\STATE{Create new Huffman element from the next byte in the stream, with the same code as $code$ and with the length $cL$}
		\STATE{Add Huffman element to table}
		\STATE{$e - 1$}
		\STATE{$code + 1$}
	\ENDFOR
\RETURN $Huffman Table$
\end{algorithmic}
\end{algorithm}
